\documentclass[11pt, a4paper]{article}

% Packages
\usepackage[utf8]{inputenc}
\usepackage[english]{babel}
\usepackage{geometry}
\geometry{a4paper, margin=1in}
\usepackage{graphicx}
\usepackage{float}
\usepackage{booktabs}
\usepackage{amsmath}
\usepackage{hyperref}
\usepackage{enumitem}
\usepackage{caption}
\usepackage{subcaption}

% Title and author information
\title{\textbf{Machine Learning Course} \\ Dataset Description Report}
\author{
    Scherling Christopher (12119060) \\
    Kohl Leonhard (12047036) \\
    Campbell Lucas (12536848) \\
    \textit{Group 68}
}
\date{\today}

\begin{document}

\maketitle

\section{Introduction and Dataset Selection}

In this report, we describe two datasets selected for our machine learning analysis: the \textbf{Road Safety Dataset} and the \textbf{Adult (Census Income) Dataset}. These datasets were chosen to satisfy the requirement of having different characteristics across multiple dimensions.


\section{Dataset 1: Road Safety}


\section{Dataset 2: Adult}

The Adult/Census Income dataset is a binary classification problems where we have to determine if a person earns >50.000\$ per year or under.
It was choosen because it provides real-worls socialeconomic relevance and offers diverse attributes (categorical and numerical) and also requires preprocessing.

\subsection(Dataset structure)
The dataset looks as follows:
\begin{itemize}[noitemsep]
    \item 48.842 instances
    \item 15 features (14 input attributes + 1 target)
    \item 2 classes (binary classification: >50.000\$, <50.000\$)
\end{itemize}
The attributes are split as 6 numeric attributes (age, fnlwgt, education-num, capital-gain, capital-loss, hours-per-week) 
and 9 nominal attributes (workclass, education, marital-status, occupation, relationship, race, sex, native-country, class(target)).




\subsection



\end{document}
