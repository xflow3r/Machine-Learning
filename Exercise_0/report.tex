\documentclass[11pt, a4paper]{article}

% Packages
\usepackage[utf8]{inputenc}
\usepackage[english]{babel}
\usepackage{geometry}
\geometry{a4paper, margin=1in}
\usepackage{graphicx}
\usepackage{float}
\usepackage{booktabs}
\usepackage{amsmath}
\usepackage{hyperref}
\usepackage{enumitem}
\usepackage{caption}
\usepackage{subcaption}

% Title and author information
\title{\textbf{Machine Learning Course} \\ Dataset Description Report}
\author{
    Scherling Christopher (12119060) \\
    Kohl Leonhard (12047036) \\
    Campbell Lucas (12536848) \\
    \textit{Group 68}
}
\date{\today}



\begin{document}

\maketitle

\newpage

\section{Introduction and Dataset Selection}

In this report, we describe two datasets selected for our machine learning analysis: the \textbf{Road Safety Dataset} and the \textbf{Adult (Census Income) Dataset}. These datasets were chosen to satisfy the requirement of having different characteristics across multiple dimensions.


\section{Dataset 1: Road Safety}

The Road Safety (TODO: add hyperlink here with the OpenML dataset ID) dataset contains information about road accidents in the UK from 1979 to 2015. It was chosen for its rich feature set, including categorical and numerical data, and its relevancy for traffic safety analysis.

\subsection{Dataset structure}
The dataset contains:
\begin{itemize}[noitemsep]
    \item 67 features
    \item 363,243 instances
\end{itemize}
Given the large number of features, we will focus on a subset of relevant attributes for our analysis, including:
\begin{itemize}[noitemsep]
    \item Age\_of\_Driver: ratio attribute
    \item Age\_of\_Casualty: ratio attribute
    \item 1st\_Point\_of\_Impact: nominal attribute
    \item Sex\_of\_Driver: nominal attribute
    \item Sex\_of\_Casualty: nominal attribute
    \item Weather\_Conditions: nominal attribute
    \item Road\_Surface\_Conditions: nominal attribute
\end{itemize}

\subsection{Target Attribute}
The target attribute for this dataset is the accident severity. This is a categorical variable with three possible values: fatal, serious, and slight.
TODO: add more details about the target attribute, such as its distribution and importance. ALSO check actual severity value labels once dataset is loaded.

\subsection{Data Preprocessing}
TODO describe preprocessing steps, such as handling missing values, encoding categorical variables, and normalizing numerical features (including describing data ranges).

\newpage
\section{Dataset 2: Adult}

The Adult/Census Income dataset is a binary classification problem where we have to determine if a person earns $>$50.000\$ per year or under.
It was chosen because it provides real-world socioeconomic relevance and offers diverse attributes (categorical and numerical) and also requires preprocessing.

\subsection{Dataset structure}
The dataset contains 48.842 instances with 15 features (14 input attributes + 1 target) and 2 classes (binary classification: $>$50.000\$, $<$50.000\$).
The attributes are split as 6 numeric attributes (age (17-90), fnlwgt (12,285-1,484,705), education-num (1-16), capital-gain (0-99,999), capital-loss (0-4,356), hours-per-week (1-99)) 
and 9 nominal attributes (workclass, education, marital-status, occupation, relationship, race, sex, native-country, class(target)), where education is also ordinal, which means the order has to be preserved.
The numeric attributes have completely different scales which means some sort of normalizing/scaling will be done.

\begin{figure}[H]
    \centering
    \begin{subfigure}[b]{0.32\textwidth}
        \centering
        \includegraphics[width=\textwidth]{images_adult/01_target_distribution.png}
        \caption{Income class distribution showing significant class imbalance (76.1\% vs 23.9\%)}
        \label{fig:target}
    \end{subfigure}
    \hfill
    \begin{subfigure}[b]{0.32\textwidth}
        \centering
        \includegraphics[width=\textwidth]{images_adult/06_missing_values.png}
        \caption{Missing values in workclass, occupation, and native-country features}
        \label{fig:missing}
    \end{subfigure}
    \hfill
    \begin{subfigure}[b]{0.32\textwidth}
        \centering
        \includegraphics[width=\textwidth]{images_adult/05_capital_gain_distribution_2.png}
        \caption{Capital-gain distribution (non-zero only)}
        \label{fig:capital}
    \end{subfigure}
    \label{fig:overview}
\end{figure}
The dataset contains significant class imbalance, as can be seen in Figure~\ref{fig:target} with (76.1\% $\leq$ 50.000\$ vs. 23.9\% $> 50.000\$ $), which could impact model evaluation accuracy.

In Figure~\ref{fig:missing} we can see that the dataset contains 6.465 missing values across the 3 features occupation, workclass and native-country. 
Also the features capital-gain and capital-loss are highly skewed and will require some sort of transformation, which can be observed in Figure~\ref{fig:capital}.

\begin{figure}[H]
    \centering
    \includegraphics[width=0.65\textwidth]{images_adult/07_education_vs_income.png}
    \caption{Relationship between education level and income, showing strong correlation between higher education and income above 50.000\$}
    \label{fig:education_income}
\end{figure}

We can also observe a strong correlation between education level and the predicted income in Figure~\ref{fig:education_income}, since people with a professional degree have a 72\% chance of earning $>50.000\$$, while people with only pre-school have a 2\% chance.
\end{document}
