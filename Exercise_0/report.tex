\documentclass[11pt, a4paper]{article}

% Packages
\usepackage[utf8]{inputenc}
\usepackage[english]{babel}
\usepackage{geometry}
\geometry{a4paper, margin=1in}
\usepackage{graphicx}
\usepackage{float}
\usepackage{booktabs}
\usepackage{amsmath}
\usepackage{hyperref}
\usepackage{enumitem}
\usepackage{caption}
\usepackage{subcaption}

% Title and author information
\title{\textbf{Machine Learning Course} \\ Dataset Description Report}
\author{
    Scherling Christopher (12119060) \\
    Kohl Leonhard (12047036) \\
    Campbell Lucas (12536848) \\
    \textit{Group 68}
}
\date{\today}

\begin{document}

\maketitle

\section{Introduction and Dataset Selection}

In this report, we describe two datasets selected for our machine learning analysis: the \textbf{Road Safety Dataset} and the \textbf{Adult (Census Income) Dataset}. These datasets were chosen to satisfy the requirement of having different characteristics across multiple dimensions.

\subsection{Rationale for Dataset Selection}

The Road Safety dataset and Adult dataset provide strong contrasts in the following areas:

\begin{itemize}[noitemsep]
    \item \textbf{Sample size}: Road Safety contains 363,243 instances while Adult contains 48,842 instances (large vs. medium)
    \item \textbf{Dimensionality}: Road Safety has 67 features while Adult has 14 features (high vs. low dimensional)
    \item \textbf{Attribute types}: Road Safety is predominantly numeric (91\%) while Adult has a balanced mix of numeric and categorical attributes
    \item \textbf{Missing values}: Road Safety contains 2,181,757 missing values while Adult has moderate missing values in specific columns
    \item \textbf{Preprocessing requirements}: Different preprocessing strategies are needed for each dataset
\end{itemize}


\section{Dataset 1: Road Safety}


\section{Dataset 2: adult}



\end{document}
