\documentclass[11pt, a4paper]{article}

% Packages
\usepackage[utf8]{inputenc}
\usepackage[english]{babel}
\usepackage{geometry}
\geometry{a4paper, margin=1in}
\usepackage{graphicx}
\usepackage{float}
\usepackage{booktabs}
\usepackage{amsmath}
\usepackage{hyperref}
\usepackage{enumitem}
\usepackage{caption}
\usepackage{subcaption}

% Title and author information
\title{\textbf{Machine Learning Course} \\ Dataset Description Report}
\author{
    Scherling Christopher (12119060) \\
    Kohl Leonhard (12047036) \\
    Campbell Lucas (12536848) \\
    \textit{Group 68}
}
\date{\today}



\begin{document}

\maketitle
\section{Introduction and Dataset Selection}

In this report, we describe two datasets selected for our machine learning analysis: the \textbf{Road Safety Dataset} and the \textbf{Adult (Census Income) Dataset}. These datasets were chosen to satisfy the requirement of having different characteristics across multiple dimensions.


\newpage


\section{Dataset 1: Road Safety}

The Road Safety dataset contains information about road accidents in the UK from 1979 to 2015. It was chosen for its rich feature set, including categorical and numerical data, and its relevancy for traffic safety analysis.

\subsection{Dataset structure}
For more details, see the Road Safety dataset on OpenML: 
\href{https://www.openml.org/d/42803}{OpenML Dataset 42803}

The dataset contains:
\begin{itemize}[noitemsep]
    \item 67 features
    \item 363,243 instances
\end{itemize}
Given the large number of features, we will focus on a subset of relevant attributes for our analysis, including:
\begin{itemize}[noitemsep]
    \item \textbf{1st\_Point\_of\_Impact}: nominal attribute (Front, Back, Offside, Nearside, No impact)
    \item \textbf{Weather\_Conditions}: nominal attribute (Fine, Rain, Snow, High Winds, Rain + High Winds, Snow + High Winds, Fog or mist, Other)
    \item \textbf{Road\_Surface\_Conditions}: nominal attribute (Dry, Wet or damp, Snow, Frost or ice, Flood over 3cm. deep, Oil or diesel, Mud)
    \item \textbf{Light\_Conditions}: nominal attribute (Daylight, Darkness - lights lit, Darkness - lights unlit, Darkness - no lighting, Darkness - lighting unknown, Data missing or out of range)
    \item \textbf{Vehicle\_Type}: nominal attribute (28 types) - May need to further preprocess into weight classes or power classes
    \item \textbf{Casualty\_Severity}: ordinal attribute (Fatal, Serious, Slight)
    \item \textbf{Day\_of\_Week}: nominal attribute
    \item \textbf{Car\_Passenger}: nominal attribute (No passenger, Front seat passenger, Rear seat passenger)
\end{itemize}

\subsection{Target Attribute}
The planned target attribute for this dataset will be the Age Band of the Driver. This is an interval variable, with possible values 0 - 5, 6 - 10, 11 - 15, 16 - 20, 21 - 25, 26 - 35, 36 - 45, 46 - 55, 56 - 65, 66 - 75, and Over 75.

\subsection{Data Preprocessing}
Depending on the individual crash report, several fields contain missing data, which will need to be addressed during preprocessing. Additionally, Vehicle\_Type is currently encoded as a categorical variable via a numeric value, but its distribution has several gaps between values, so could be adjusted to be smoother. Depending on the vehicle type, the categories could also be rearranged in terms of size and power (CC) to make the numeric encoding more meaningful.

\newpage
\section{Dataset 2: Website Phishing}
The Website Phishing dataset is a multi-class classification problem where we have to determine if a website is legitimate, suspicious, or phishing. It was chosen because it addresses a critical cybersecurity challenge in e-commerce and e-banking and provides diverse integer-encoded features representing various URL and website characteristics.
It will also require some preprocessing due to class imbalance. It is a nice contrast to the first dataset since the target attribute is nominal and a small dataset compared to an interval target attribute and a large dataset.

\subsection{Dataset structure}
The dataset contains 1.353 instances with 10 features (9 input attributes + 1 target) and 3 classes (legitimate, suspicious, phishing). 
All attributes are integer-encoded categorical features with values ranging from -1 to 1 (except having\_IP\_Address which is 0, 1): SFH (Server Form Handler), popUpWindow, SSLfinal\_State, Request\_URL, URL\_of\_Anchor, web\_traffic, URL\_Length, age\_of\_domain, having\_IP\_Address, and Result (target). 
Some features like age\_of\_domain, web\_traffic, and URL\_Length are ordinal with natural ordering, while others like having\_IP\_Address are nominal.

\begin{figure}[H]
    \centering
    \begin{subfigure}[b]{1\textwidth}
        \centering
        \includegraphics[width=\textwidth]{images_phishingWebsites/phishing_dataset_2plots.png}
    \end{subfigure}
    \caption{(Left) Class distribution. (Right) Risky feature indicators .}
    \label{fig:phishing_overview}
\end{figure}

As shown in Figure~\ref{fig:phishing_overview}, the dataset shows some class imbalance with the suspicious class representing only 7.6\% of cases, which will require some processing like class weights during training.
The risky feature analysis shows that SFH (Server Form Handler) and SSLfinal\_State are the strongest indicators for phishing sites showing 62\% and 43\% suspicious indicators respectively. Meanwhile the same features appear only around 10\% in legitimate sites. 
It is also important to note that no popUpWindow was found on phishing sites in this dataset.

\end{document}
